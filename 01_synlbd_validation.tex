\documentclass{article}
\usepackage{statrep}
\usepackage{parskip,xspace}
\usepackage{attachfile}
\newcommand*{\Statrep}{\mbox{\textsf{StatRep}}\xspace}
\newcommand*{\Code}[1]{\texttt{\textbf{#1}}}
\newcommand*{\cs}[1]{\texttt{\textbf{\textbackslash#1}}}
\setcounter{secnumdepth}{0}
\title{Example and Tutorial for SynLBD Validation}
\author{Lars Vilhuber, Jorgen Harris, and Emin Dinlersoz}
\date{\today}
\begin{document}
\maketitle
\section{Introduction}
We want to provide an example of proper validation criteria, using a fake dataset as our input.


\begin{quotation}
This article uses the \Statrep \LaTeX\ package.
The package is available
for download at \texttt{http://support.sas.com/StatRepPackage}. To generate this document, 
\begin{verbatim}
pdflatex file.tex
sas file_SR.sas
pdflatex file.tex
pdflatex file.tex
\end{verbatim}
or see Appendix~\ref{sec:statrep}.
\end{quotation}


\section{The Fake Dataset}
We create a dataset that approximates, very roughly, the characteristics of the establishment and employment distribution of a real dataset such as the SynLBD and LBD. We use this so that the document is maximally portable, given distribution restrictions for both SynLBD and LBD.

\begin{Datastep}[program]
options nocenter;
libname home ".";

/* basic parameters */
%let indcnt=40;
%let maxestabcnt=10000;
%let minestabcnt=100;
%let seed1=123456;

%let maxemp=42000;
%let minemp=1;
%let seed2=1234567;
\end{Datastep}

\begin{Datastep}
/* draw estab count distribution across industries*/
/* use log normal */
data industries;
  do industry = 1 to &indcnt.;
  estabs= exp(ranuni(&seed1.)*
                (log(&maxestabcnt.)
                -log(&minestabcnt.)) 
                + log(&minestabcnt.));
  output;
  end;
  run;


/* now draw employment for each estab in each industry */

data fakelbd;
  set industries;
  by industry;
  drop i;
  do lbdnum=100000*industry+1 to 100000*industry+estabs; 
    do year=1 to 3;
       emp= exp(ranuni(&seed2.)
         *(log(&maxemp.)-log(&minemp.)) 
         + log(&minemp.));
    payroll  = emp*30*ranuni(3153);
    output;
  end; end;
run;
\end{Datastep}

We can assess the distributions, first of establishments:

\begin{Sascode}[store=univarA,program]
ods graphics on;
  proc univariate data=industries;
  var estabs;
  run;
  ods graphics off;
 \end{Sascode}
\Listing[store=univarA,caption={Statistics on establishments},objects=BasicMeasures ]{univarA}

Let's have a look at the distribution of employment:

\begin{Sascode}[store=univarB,program]
ods graphics on;
proc univariate data=fakelbd;
var emp;
run;
ods graphics off;
\end{Sascode}
\Listing[store=univarB,caption={Statistics on employment},objects=BasicMeasures]{univarB}

The number of establishments across industries varies, which will lead to difficulties if we want to obtain results for certain industries:

\begin{Sascode}[store=obsperind,program]
proc freq data=fakelbd;
table industry;
run;
\end{Sascode}
\Listing[store=obsperind,caption={Number of obs per industry}]{obsperind}

\section{Project 1:  Analysis that meets validation requirements}
This example is a project where the analysis, and the validation request, meet the requirements. This project is interested in ... First, the researcher prepares the data:

\begin{Datastep}
/*Prepare data*/
/* program: 01_prepdata.sas */
data analysis1;
set fakelbd;
by industry lbdnum year;
wage = payroll/emp;
if first.lbdnum then do; 
	lagE = .;         
	lagp = .;         
	lagw = .;         
end;
else                 do; 
	lagE = lag(emp); 
	lagp=lag(payroll); 
	lagw = lag(wage); 
end;
empgrowth = emp/lage;
wagegrowth= wage/lagw;
run;
\end{Datastep}

Then, the regression of interest to the researcher is run:

\begin{Sascode}[store=regA]
/*Regression of interest*/
proc reg data=analysis1;
by industry;
where industry le 2;
model empgrowth = lagE lagw;
output out=obsds1 r=inc;
ods output parameterestimates=param1;
run;
ods trace off;
\end{Sascode}
The result of the regression is the following output (here for the first industry only):

\Listing[store=regA,objects=Reg.ByGroup1.MODEL1.Fit.empgrowth.ParameterEstimates,caption={Project 1: Parameter estimates}]{regAparms}

In order to prepare for validation and disclosure avoidance review of the \textit{confidential} analysis, the researcher must determine  the effective sample size of each parameter in terms of establishments and total observations. Ideally, this is provided as an ``augmented'' results table that allows the Census Bureau disclosure officer to assess the whole picture. The following code will generate that information:
\begin{Datastep}
proc sql;
create table discreview1 as
select industry,count(distinct lbdnum) as nEstabs,count(*) as nObs
from obsds1
where inc ne .
group by industry
;quit;
data discreview1;
  merge discreview1(in=_a)
               param1(in=_b);
   by industry;
 run;
\end{Datastep}

Finally, in order to prepare the validation request, as well as the release request for the synthetic data results, \textit{both} tables are written out as CSV files:
\begin{Datastep}
/*Export validation table and sample size table*/
proc export data=param1 file="./validationtable1.csv" dbms=csv replace;
run;
\end{Datastep}
\begin{Sascode}[store=paramAcsv,program]
proc print data=param1;
run;
\end{Sascode}
\Listing[store=paramAcsv,label=tab:paramA]{paramAcsv}

\begin{Datastep}
proc export data=discreview1 file="./discreview1.csv" dbms=csv replace;
run;
\end{Datastep}
\begin{Sascode}[store=discreviewA,program]
proc print data=discreview1;
run;
\end{Sascode}
\Listing[store=discreviewA,label=tab:discA]{discreviewA}

In fact, if using \LaTeX, the researcher could attach all programs and output from the synthetic data to the validation request, and submit it:

\begin{itemize}
\item 01\_synlbd\_validation\_SR.sas \attachfile{01_synlbd_validation_SR.sas}
\item validationtable1.csv \attachfile{validationtable1.csv}
\item discreview1.csv \attachfile{discreview1.csv}
\end{itemize}

Note that the result tables as shown here would be based on the synthetic data, and both \Code{discreview1.csv} and \Code{validationtable1.csv} would be released to the researcher. However, the results validated against the confidential data would differ from those reported in Figure~\ref{tab:paramA}, and the  \Code{discreview1.csv} file generated from the confidential data, using the code submitted by the researcher, would NOT be released.





\newpage

\appendix
\section{Appendix: How to compile a StatRep document\label{sec:statrep}}

When you use the \Statrep \LaTeX\ package, you use the following
four-step process to create an executable document that
enables you to ensure that your research results are reproducible:
\begin{enumerate}
\item Create your \LaTeX\ document so that it contains your text,
data, and SAS code.

\item Compile your document with pdf\LaTeX\ to generate the SAS
program.

\item Execute the SAS program to capture your output. For each
code block in your document, SAS creates a SAS Output Delivery System (ODS)
document that contains the resulting output.

For each output request in your document, SAS replays the
specified output objects to external files. All your
requested output is generated and captured when you execute
the generated SAS program.

\item Recompile your \LaTeX\ document. In this step, the
requested outputs are embedded in the resulting final PDF document.

You might need to repeat this step so that \LaTeX\ can measure the
listing outputs to ensure that they are framed appropriately.
\end{enumerate}


\end{document}
